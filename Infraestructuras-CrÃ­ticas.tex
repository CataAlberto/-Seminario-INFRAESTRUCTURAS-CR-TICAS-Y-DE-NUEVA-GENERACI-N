\documentclass{article}

\usepackage{arxiv}

\usepackage[utf8]{inputenc} % allow utf-8 input
\usepackage[T1]{fontenc}    % use 8-bit T1 fonts
\usepackage{lmodern}        % https://github.com/rstudio/rticles/issues/343
\usepackage{hyperref}       % hyperlinks
\usepackage{url}            % simple URL typesetting
\usepackage{booktabs}       % professional-quality tables
\usepackage{amsfonts}       % blackboard math symbols
\usepackage{nicefrac}       % compact symbols for 1/2, etc.
\usepackage{microtype}      % microtypography
\usepackage{graphicx}

\title{Infraestructuras críticas Sector químico}

\author{
    Catalina ALberto
   \\
    Ingeniería UnCUYO \\
   \\
  \texttt{\href{mailto:catavalberto@gmail.com}{\nolinkurl{catavalberto@gmail.com}}} \\
   \And
    Dalia Orlinski
   \\
    Ingeniería UnCUYO \\
   \\
  \texttt{\href{mailto:daliaorlinski@gmail.com}{\nolinkurl{daliaorlinski@gmail.com}}} \\
   \And
    Victoria SIlva
   \\
    Ingeniería UnCUYO \\
   \\
  \texttt{\href{mailto:victoriasilva201101@gmail.com}{\nolinkurl{victoriasilva201101@gmail.com}}} \\
   \And
    Andrés Soria
   \\
    Ingeniería UnCUYO \\
   \\
  \texttt{\href{mailto:andressoria1221@gmail.com}{\nolinkurl{andressoria1221@gmail.com}}} \\
   \And
    Macarena Vacas
   \\
    Ingeniería UnCUYO \\
   \\
  \texttt{\href{mailto:macarenavacas2014@gmail.com}{\nolinkurl{macarenavacas2014@gmail.com}}} \\
  }


% tightlist command for lists without linebreak
\providecommand{\tightlist}{%
  \setlength{\itemsep}{0pt}\setlength{\parskip}{0pt}}



\begin{document}
\maketitle


\begin{abstract}

\end{abstract}


\hypertarget{introducciuxf3n}{%
\section{Introducción}\label{introducciuxf3n}}

\label{sec:headings}

En el siguiente informe se desarrollará un breve resumen del tema
Infraestructuras críticas, con la información proporcionada en el
seminario de ``INFRAESTRUCTURAS CRÍTICAS Y DE NUEVA GENERACIÓN'' dictado
por el profesor Gustavo Masera. En específico se abordará uno de los 16
sectores que CISA considera como Infraestructuras críticas, el Sector
químico.

\hypertarget{infraestructuras-cruxedticas}{%
\section{Infraestructuras críticas}\label{infraestructuras-cruxedticas}}

\label{sec:others}

Antes de centrarnos en el desarrollo es necesario definir que es una
Infraestructura y que hace que la misma sea crítica.

Infraestructura se refiere al conjunto de instituciones, redes y
sistemas que son fundamentales para el funcionamiento de la sociedad.
Proporcionan la base para que las actividades económicas y políticas
puedan desarrollarse de una manera eficiente. Las infraestructuras se
crean y funcionan en la intersección de distintas áreas profesionales,
es importante unir los distintos agentes de la sociedad (Privados,
públicos, etc) con el objetivo de mejorar las mismas y así lograr mayor
productividad, mejor acceso a los mercados internacionales y una mayor
protección de los bienes públicos.

Se pueden clasificar las infraestructuras según sus características o su
finalidad, dentro de la primera existe una clasificación llamada
``Infraestructuras críticas''. Estas son aquellas que se consideran
esenciales para el funcionamiento de una sociedad, tanto así que una
perturbación o destrucción afectaría gravemente a la seguridad nacional,
la salud pública, la economía o el bienestar de la población.

\hypertarget{sector-quuxedmico}{%
\section{Sector químico}\label{sector-quuxedmico}}

El sector químico se posiciona como un gigante dentro de la economía de
los EEUU, generando gran impacto en distintos aspectos. Es el sector con
mayor porcentaje de exportación y es una gran fuente de generación de
empleo.

Éste se encarga de convertir más de 700.000 materias primas en productos
finales para diversos sectores. Existen cuatro componentes principales

\begin{itemize}
\item
  Químicos básicos: La base fundamental para la elaboración de otros
  productos químicos.
\item
  Productos químicos especiales: Utilizados en industrias específicas
  como la farmacéutica, la cosmética y la electrónica.
\item
  Químicos agrícolas: Indispensables para la producción agrícola y el
  cuidado de los cultivos.
\item
  Productos de consumo: Presentes en nuestro día a día, desde artículos
  de limpieza hasta materiales de construcción.
\end{itemize}

Los productos químicos de consumo, en particular, generan una
dependencia crítica de los sectores de infraestructura, lo que hace que
la industria química sea un componente esencial de la seguridad nacional
y económica.

El sector químico está sujeto a diversos problemas transversales, tales
como infraestructura de transporte obsoleta, dependencia con otros
sectores, accidentes industriales, entre otros. Estos pueden generar
interrupción en las cadenas de suministro, aumento en los gastos de
capital, pérdida de información sensible de seguridad y operaciones, y
tener otros impactos graves. Reconocerlos e intentar mitigarlos es
crucial para un desarrollo eficiente.

A continuación se presenta un breve resumen de los distintos problemas y
como afectan a la sociedad

\begin{itemize}
\tightlist
\item
  Infraestructura de transporte obsoleta
\end{itemize}

El sector químico requiere transporte adecuado para operar de manera
efectiva y segura. La antigüedad y el mal estado de carreteras, puentes,
calles y puertos crean una mayor vulnerabilidad de las infraestructuras
a las interrupciones, lo que puede generar retrasos.

En el 2017 la Sociedad Americana de Ingenieros Civiles realizó un
informe que calificó la Infraestructura de transporte de los Estados
Unidos con una D+, lo que significa que este sector Unidos presenta
importantes deficiencias que afectan negativamente su funcionamiento y
seguridad. Para mejorar esta calificación se deberían tomar medidas
tales como aumento del gasto en estas infraestructuras, establecer
políticas de mantenimiento a largo plazo, aplicación de nuevas
tecnologías, entre otras.

\begin{itemize}
\tightlist
\item
  Dependencia en otros sectores
\end{itemize}

El sector químico utiliza continuamente los recursos de agua y energía.
También depende de sectores tales como Comunicaciones, Tecnologías de la
Información, Servicios Financieros y Sistemas de Transporte. Cualquier
interrupción, falla o destrucción de alguno de estos generaría un
impacto negativo en la industria química. A su vez los productos
químicos son esenciales para el desarrollo de otros sectores, incluyendo
las industrias alimenticia, agrícola, de la salud, sistemas de agua,
entre otros. Es decir que una interrupción en el mismo generaría un gran
impacto en la sociedad.

Se destaca la interdependencia de las infraestructuras críticas de una
sociedad, el estado debe contar con una capacidad de anticipación y
recuperación; con planes de protección y redes de alerta temprana. Así
podrá lograr mitigar las fallas en alguna infraestructura que luego
podría afectar a las demás.

\begin{itemize}
\tightlist
\item
  Accidentes industriales
\end{itemize}

Si bien el sector químico opera con un alto nivel de seguridad, los
accidentes pueden ocurrir. Datos mes a mes de CSAC indican que los
accidentes industriales representan un tercio de los incidentes químicos
reportados. Los reportes indican que estos incidentes generalmente
involucran almacenamiento, transporte, producción y laboratorios.

Los accidentes pueden ocurrir ya sea por errores humanos o por procesos
defectuosos, se detectó que existen equipos obsoletos y falta de
mantenimiento preventivo. Es importante que las empresas entiendan el
peso que tiene el sector químico en la sociedad y que un error puede
generar grandes consecuencias, por esto se deben llevar a cabo
mantenimientos periódicos así como inspecciones reglamentarias por el
lado del estado con el objetivo de prevenir accidentes irremediables.

\hypertarget{conclusiuxf3n}{%
\section{Conclusión}\label{conclusiuxf3n}}

Con la información brindada en el seminario y las lecturas de distintos
reportes de CISA, se pudo comprender la importancia de las
infraestructuras críticas para un desarrollo eficiente de la sociedad.
Es fundamental que los diferentes actores colaboren estrechamente para
lograr mejoras en los aspectos sociales, económicos y medioambientales.

Se comprendió la importancia del sector químico para diversos sectores
de la sociedad, y cómo distintos problemas podrían afectar a estos
sectores. Por lo tanto, es crucial desarrollar planes de contingencia y
alerta temprana para mitigar estos riesgos.

\hypertarget{referencias}{%
\section{Referencias}\label{referencias}}

\href{https://www.cisa.gov/topics/critical-infrastructure-security-and-resilience/critical-infrastructure-sectors/chemical-sector}{CISA}

\bibliographystyle{unsrt}
\bibliography{references.bib}


\end{document}
